\chapter{Brownian Motion}
\begin{defi}
Let $(E, \Sigma)$ be a measurable space and $T$ a set.
A collection of $(E, \Sigma)$-valued random variables (RVs)
$\X = (X_t)_{t\in T}$ is called \defn{$E$-valued stochastic process (SP) with index set $T$}.
\end{defi}
%TODO Kasten

\begin{bsp}
\begin{enumerate}[label=(\alph*)]
\item An SP  with $(E, \Sigma) = (\Rd, \Borel(\Rd))$
	and $T=\R_{\geq 0}$
	is called \defn{$\Rd$-valued, continuous-time SP}.
\item For $E = \{-1,1\}$, \(\Sigma = \Po(E)\) and $T=\Z^d$,
	the SP is called \defn{spin system}.
\item If $E$ is countable and $T=\N_0$, we speak of a \defn{time discrete SP}.
\end{enumerate}
\end{bsp}

From a dynamical point of view,
$X_t$ is a $t$-dependent quantity that changes with time.
This perspective is suitable for the comprehension of the first and third example.
From a global point of view,
an SP is a single RV with values in the space $\Omega=E^T=\{f\colon T\to E\}$.
In the first example,
this means to consider the whole \defn{path} $(X_t)_{t\geq 0}$ as one object.
In the second example,
each spin configuration in $\{-1,1\}^{\Z^d}$ is an element of $\Omega$.
This raises some questions:
What is the “right” $\sigma$-algebra on \(\Omega\)?
Does it even exist?

\begin{defi}
Let $\X = (X_t)_{t\in T}$ be an SP
where the state space $E$ is a group, \eg $E=\Rd$,
and $T\subseteq \R$.
The family $(X_{s,t})_{s,t\in T}$ with $X_{s,t}\coloneqq X_t-X_s$
is called the \defn{increment process of $\X$} or \defn{set of increments}.

An SP has \defn{independent increments}
if for all $n\in \N$ and all
$s_1 < t_1 \leq s_2 < t_2 \leq \dots \leq s_n < t_n$
with $s_i, t_i \in T$,
the RVs $(X_{s_i,t_i})_{1\leq i \leq n}$ are independent.

An SP has \defn{stationary increments}
if for all $n\in \N$, all \(r \in T\) and all
$s_1 < t_1 \leq s_2 < t_2 \leq \dots \leq s_n < t_n$
with $s_i,t_i \in T$,
we have
\begin{align*}
(X_{s_i,t_i})_{i=1,\dots,n}\sim  (X_{s_i+r,t_i+r})_{i=1,\dots,n},
\end{align*}
\ie that the increments are equal in distribution
regardless at which time we started looking.
\end{defi}
%TODO Kasten

\begin{defi}
An $\Rd$-valued SP $\B=(B_t)_{t \in \R_{\geq 0}}$
is called \emph{Brownian Motion} (BM) if
\begin{enumerate}[label=(B\arabic*)]
\item $B_0(\omega)=0$ for almost all $w\in \Omega$,
\item $\B$ has independent increments,
\item $\B$ has stationary increments,
\item the increments are normally distributed, \ie
	\[B_{s,t} \coloneqq B_t - B_s \sim B_{t-s}\sim \Ndist{0}{(t-s)\Id},\]
	and
\item the map $t\mapsto B_t(\omega)$ is continuous for all $\omega \in \Omega$.
\end{enumerate}
\end{defi}
%TODO Kasten

\begin{bem}
Checking the requirements (B0)\,--\,(B4)
in view of $B_t=\int_0^t \xi_s~\mathrm{d}s$:
\begin{enumerate}[label=(B\arabic*)]
\item $\int_0^0\xi_s~\mathrm{d}s=0$
\item $B_t-B_s=\int_s^t \xi_r~\mathrm{d}r$
	and $\xi_r \amalg (\xi_s)_{s\not =r}~~\forall r$
% TODO: Is this correct?
%	and the \(\xi_r\) are independent from each other
\item The distribution of $\xi_r$ does not depend on $r$.
\item Central limit theorem and Riemann approximation.
\item The map $t \mapsto \int_0^t f_s~\mathrm{d}s$ is continuous
	for all “sensible” functions $f$,
	in particular for $f = \xi$.
\end{enumerate}
\end{bem}

\begin{defi}
The \defn{Gaussian measure} $\Ndist{m}{\sigma^2}$
with mean $m$ and variance $\sigma^2$ is
the probability measure on $(\R,\mathcal{B}(\R))$ with Lebesgue density
\begin{align*}
	g_{m, \sigma^2}(x) = \frac{1}{\sqrt{2\pi \sigma^2}} \exp\left(-\frac{1}{2\sigma^2}(x-m)^2\right).
\end{align*}
\end{defi}

\begin{prop}
Let $X\sim \Ndist{m}{\sigma^2}$.
Then:
\begin{enumerate}[label=(\alph*)]
\item $\E{X } =m$, $\V{X} = \sigma^2$
\item We have the \defn{Gaussian tail estimate}
	% TODO: Refactor terms?
	\begin{align*}
	\frac{1}{\sqrt{2\pi}} \frac{C}{C^2+1}\e^{-\frac{C^2}{2}}
	\leq
	\mathds{P}(X-m\geq C \sigma)
	\leq
	\frac{1}{\sqrt{2\pi}} \frac{1}{C}\e^{-\frac{C^2}{2}},
	\end{align*}
	for all $C>0$, $\sigma>0$.
\item For $(m_k)_{k\in \N}\subseteq \R$, $m\in \R$,
	$(\sigma_k)_{k\in \N}\subseteq \R_{\geq 0}$ and $\sigma\in \R_{\geq 0}$
	we have that
	$(m_k, \sigma_k) \to (m, \sigma)$
	if and only if
	$\Ndist{m_k}{\sigma_k^2} \distto  \Ndist{m}{\sigma^2}$.
\end{enumerate}
\end{prop}

\begin{defi}
An $\Rd$-valued RV $X$
is called \defn{$d$-dimensional Gaussian}
if for all linear functionals $L\colon \Rd\to \R$
there are $m$, $\sigma^2$ with $LX \sim \Ndist{m}{\sigma^2}$.
Explicitly:
If $\X=(X^1,\dots,X^d)$,
this means that for all $a_1,\dots,a_d \in \R$
there are $m$, $\sigma^2$
such that $\sum_{i=1}^d a_i X^i \sim \Ndist{m}{\sigma^2}$.
\end{defi}

\begin{bsp}
\begin{enumerate}[label=(\alph*)]
\item If $X^1,\dots,X^d$ are independent $1$-dimensional Gaussian,
	then $\textbf{X}=(X^1,\dots,X^d)$ is $d$-dimensional Gaussian.
\item \emph{Warning:}
	Without independence, this is not true in general.
	Consider $X^1\sim \Ndist01$ and
	\begin{align*}
	X^2(\omega) = \begin{cases}
		-X^1(\omega), &\text{if \(\abs{X^1(\omega)}\leq 1\)},\\
		+X^1(\omega), &\text{if \(\abs{X^1(\omega)}>1\)}.
	\end{cases}
	\end{align*}
	Then, $X^2\sim \Ndist01$
	(to check this, compute $\mathds{P}(X^2<c)$ for all $c\in \R$)
	but $(X^1,X^2)$ is not Gaussian as
	$\abs{X^1(\omega)-X^2(\omega)}\leq 2$ for all $\omega \in \Omega$ and
	$\abs{X^1(\omega)-X^2(\omega)}\not \equiv 0$,
	which implies that $X^1-X^2$ is not Gaussian.
\end{enumerate}
\end{bsp}

\begin{ex}
	Are there pairwise independent
	\(X^1, \dots, X^d \sim \Normaldistribution\)
	such that \(\X = (X^d, \dots, X^d)\) is not Gaussian?
\end{ex}

\begin{prop}
A real RV $X$ is $\Ndist{m}{\sigma^2}$-distributed if and only if
its characteristic function is given by
\begin{align}\label{eq:characteristicfunction}
	\varphi_X(u)
	= \e^{\i um}\e^{-\frac{1}{2}\sigma^2u^2}
	= \exp\autol(\i um - \frac12\sigma^2u^2\autor).
\end{align}

\begin{proof}
Recall that $\varphi_X(u) = \E[norm]{\e^{\i uX}}$ uniquely determines
the distribution of $X$.
So it is enough to show \eqref{eq:characteristicfunction}
for $X \sim \Ndist{m}{\sigma^2}$.
Since we have the regularity
$\varphi_{X+m}(u) = \E[norm]{\e^{\i u(X+m)}}=\e^{\i um}\varphi_X(u)$,
it suffices to consider the case $m=0$.

By the Lebesgue differentiation theorem we have
% FIXME: I do not see these --Peter
\begin{align*}
\frac{\mathrm{d}}{\mathrm{d}u}\varphi_X(u)&=\frac{1}{\sqrt{2\pi \sigma^2}} \int \i x\e^{\i ux}\e^{-\frac{x^2}{2 \sigma^2}}~\mathrm{d}x\\
&=\frac{1}{\sqrt{2 \pi \sigma^2}} \int \i (\i u \e^{\i ux})\sigma^2 \e^{-\frac{x^2}{2 \sigma^2}}~\mathrm{d}x\\
&=-u \sigma^2 \varphi_X(u),
\end{align*}
and $\varphi_X(0)=1$.
Hence, $h(u)=\ln(\varphi_X(u))$ solves the ODE
\begin{align*}
	h'(u) &= \frac{\varphi_X'(u)}{\varphi_X(u)}=-u \sigma^2\\
	h(0)  &= 0,
\end{align*}
which implies $h(u)=-\frac{1}{2}u^2\sigma^2$.
\end{proof}
\end{prop}

\begin{cor}
For $X \sim \Ndist{0}{\sigma^2}$ and $J\in \C$ we have $\E{\e^{JX}})=\e^{\sigma^2 \frac{J^2}{2}}$.
\begin{proof}
This follows by analytic continuation of the previous proposition.
\end{proof}
\end{cor}

\begin{thm}
Let $X$ be $d$-dimensional Gaussian.
\begin{enumerate}[label=(\alph*)]
\item The distribution of $X$ is uniquely determined by the \emph{mean vector of $X$} $\textbf{m}=\E{X}=\E{(X^i)}_{1\leq i\leq d} \in \Rd$ and the \emph{covariance matrix of $X$} $C=(C_{ij})_{1\leq i,j\leq d}$ with $C_{ij}=\Cov(X^i,X^j)$.
We write $X \sim \Ndist{\textbf{m}}{C}$.
\item If $C$ is invertible, then the distribution of $X$ has a Lebesgue-density and
\begin{align*}
\mathds{P}(X \in \mathrm{d}\textbf{x}))=\frac{1}{(2 \pi)^{\frac{d}{2}}} \frac{1}{(\det C)^{\frac{1}{2}}} \e^{-\frac{1}{2}(\textbf{x}-\textbf{m},C^{-1}(\textbf{x}-\textbf{m}))}~\mathrm{d}\textbf{x}.
\end{align*}
\end{enumerate}
\begin{proof}
\begin{enumerate}[label=(\alph*)]
\item Assume $X,Y$ are $d$-dimensional Gaussian with mean $\textbf{m}$ and covariance matrix $C$.
Let $a\in \Rd$, $Z=\sum_{i=1}^d a_iX^i$, $W=\sum_{i=1}^d a_i Y^i$.
Then, $Z,W$ are $1$-dimensional Gaussian with $\E{Z}=\E{W}=(a,\textbf{m})$ and
\begin{align}\label{eq:equalityvariance}
\mathds{V}(Z)=\mathds{V}(W)=(a,Ca).
\end{align}
So,
\begin{align*}
\varphi_{\textbf{X}}(a)=\E{\e^{\i (a,X)}}=\e^{\i (a,\textbf{m})}\e^{-\frac{1}{2}(a,Ca)}=\varphi_\textbf{Y}(a)
\end{align*}
holds for all $a\in\Rd$ which implies $\textbf{X}\sim \textbf{Y}$.
\item By \eqref{eq:equalityvariance}, $C$ must be positive semidefinite. If $C$ is invertible, $C$ must be positive definite.
Hence, the density is well-defined.
To check that it is the right one, compute its characteristic function (remains as an exercise). \qedhere
\end{enumerate}
\end{proof}
\end{thm}
%TODO Kasten

\begin{prop}
Let $X \sim \Ndist{\textbf{m}}{C}$ $d$-dimensional Gaussian, $A \in \R^{n\times d}$.
Then $AX\sim \Ndist{A\textbf{m}}{ACA^\ast}$ where $A^\ast$ denotes the transpose of $A$.
\begin{proof}
The proof remains as an exercise.
\end{proof}
\end{prop}

\begin{prop}
Let $\textbf{X}\sim \Ndist{\textbf{m}}{C}$.
Then $X^1,\dots,X^d$ are independent RVs if and only if $C_{ij}=0$ for all $i\not = j$, i.e., $X^i,X^j$ are uncorrelated.
\begin{proof}
\glqq $\Rightarrow$\grqq\, always holds (if the variances exist).
For the other direction let $Y^1,\dots,Y^d$ be independent with $Y^i\sim \Ndist{m_i}{C_{ii}}$.
Then by (1.13) $\textbf{X} \sim \textbf{Y}$ which implies that $(X^i)$ are independent.
%TODO label richtig machen
\end{proof}
\end{prop}

\begin{defi}
Let $(X_t)_{t \in T}$ be an $E$-valued SP defined on a probability space $(\Omega,\mathcal{F},\Prob)$. The set of \emph{finite dimensional distributions (fdd)} of $\textbf{X}$ is the family of probability measures
\begin{align*}
\{p_{t_1,\dots,t_n}\mid t_1,\dots,t_n, t_i\not =t_j \text{ if } i \not = j,n\in \N\}
\end{align*}
where $p_{t_1,\dots,t_n}=\Prob \circ (X_{t_1},\dots,X_{t_n})^{-1}$ is the image of $\Prob$ under $(X_{t_1},\dots,X_{t_n})$.
In order words, $p_{t_1,\dots,t_n}(A_1\times \dots \times A_n)=\Prob(X_{t_1}\in A_1,\dots,X_{t_n}\in A_n)$ for all \glqq good\grqq\, sets $A_1,\dots,A_n$.
\end{defi}
%Kasten
%TODO good=measurable wrt right sigma algebra

\begin{bsp}
Let $T=\N$, $E=\Z$ and $(X_n)_{n\in \N}$ be a simple random walk, this is to say
$X_n=\sum_{i=1}^n Z_i$ with $Z_i$ iid, $\Prob(Z_i=\pm 1)=\frac{1}{2}$.
Then
\begin{align*}
p_{1,4,17}(A\times B\times C)=\Prob(X_1\in A,X_4\in B,X_{17}\in C).
\end{align*}
\end{bsp}

\begin{prop}
Let $\textbf{X}$ be as in (1.16). Then its fdd fulfil the \emph{consistency conditions} that for all $t_1,\dots,t_n \in T$, $C_1,\dots,C_n \in \mathcal{E}$, $\sigma\in S_n$ it holds that
\begin{enumerate}
\item[(C1)] $p_{t_1,\dots ,t_n}(C_1\times \dots \times C_n)=p_{t_{\sigma(1)},\dots ,t_{\sigma(n)}}(C_{\sigma(1)}\dots ,C_{\sigma(n)})$,
\item[(C2)] $p_{t_1,\dots , t_n}(C_1\times \dots \times C_{n-1}\times E)=p_{t_1,\dots ,t_{n-1}}(C_1\times \dots \times C_{n-1})$.
\end{enumerate}
\begin{proof}
This remains as an easy exercise.
\end{proof}
\end{prop}
%Kasten
%TODO 1.16 ordentlich machen

\begin{defi}
An $\Rd$-valued process $(X_t)_{t \in T}$ is called \emph{Gaussian process} if all its fdd are Gaussian measures.
\end{defi}
%Kasten

\begin{bem}
\begin{enumerate}[label=(\alph*)]
\item Explicitly, $p_{t_1,\dots , t_n}$ is Gaussian on $\R^{dn}$.
\item (1.10b) shows that there are processes where $X_t$ is Gaussian for all $t\in T$ but where $\textbf{X}$ is not a Gauss process. Take for example $T=\{1,2\}$, $E=\R$, $X_1=X^1,X_2=X^2$ in (1.10).
Morale: the one-dimensional distributions are not enough to make a process Gaussian!
\item If $\textbf{X}$ is a Gauss process, its fdd are fully determined by mean function $T \to \Rd, t \mapsto \E{X_t}$ and the covariance function $T^2\to \R^{d\times d}, (s,t) \mapsto \Cov(X_s,X_t)$.
This follows immediately from Theorem 1.13
%TODO 1.13 ordentlich machen
%TODO vlt in align machen, dann sieht es schoener aus.
\end{enumerate}
\end{bem}
%TODO 1.10 ordentlich machen

\begin{thm}
\begin{enumerate}[label=(\alph*)]
\item An $\Rd$-valued Brownian motion $\textbf{B}$ is a Gaussian process with $\E{B_t}=0$ for all $t$.
%TODO bereich von t?
and 
\begin{align*}
Cov(B_s,B_t)=\E{B_s\otimes B_t}=\E{\left(B_s^iB_t^j\right)_{i,j=1,\dots,d}}=\min\{s,t\} \cdot I_{\Rd}
\end{align*}
%TODO da muss noch iwo transposed hin, hier sollte es richtig sein
\item Conversely, any Gaussian process with the mean and covariance functions from a) is a Brownian motion if it fulfills (B4).
\end{enumerate}
\begin{proof}
\begin{enumerate}[label=\alph*)]
\item Let $t_1,\dots,t_n\in \R_0^+$ with $t_1<\dots<t_n$. Then
\begin{align*}
\left(B_{t_1}(\omega),\dots,B_{t_n}(\omega)\right)^\top=\\
A\left(B_{t_1}(\omega)-B_0(\omega),B_{t_2}(\omega)-B_{t_1}(\omega),\dots,B_{t_n}(\omega)-B_{t_{n-1}}(\omega)\right)^\top
\end{align*}
%TODO first line flushleft, second flushright
with the lower triangle matrix
\begin{align*}
A=\begin{pmatrix}
1 & \cdots & 0\\
\vdots & \ddots & \vdots\\
1 & \cdots & 1
\end{pmatrix}
\end{align*} 
holds.
By (B1),(B3) and (1.9), $(B_{t_i}-B_{t_{i-1}})_{1 \leq i \leq n}\sim \Ndist{0}{C}$
with $C_{ij}=\delta_{ij}(t_i-t_{i-1})$.
By (1.13), $p_{t_1,\dots,t_n}\sim \Ndist{0}{ACA^\ast}$ which implies that $\textbf{B}$ is a Gaussian process.
Now, we compute the covariance and assume $s<t$. Then we have
\begin{align*}
\Cov(B_s,B_t)&=\E{B_s\otimes B_t}=\E{B_s\otimes (B_t-B_s)} + \E{B_s \otimes B_s}\\
&=s \Id=\min\{s,t\}\Id.
\end{align*}
\item We check that (B0)-(B2) hold, as (B3),(B4) holdn by assumption.
(B0) follows from $\V{B_0}=0$ and $\E{B_0}=0$.
For (B1) and (B2) let $0<t_1<\dots<t_n$.
The covariance matrix $(B_{t_1},\dots,B_{t_n})$ is
\begin{align*}
M=(m_{ij})_{i,j\in \N}=(t_{\min\{i,j\}})_{i,j \in \N}
\end{align*}
and with $A$ as in a).
Then, $(B_{t_1}-B_0,B_{t_2}-B_{t_1},\dots,B_{t_n}-B_{t_{n-1}})$ has covariance matrix
\begin{align*}
M'=A^{-1}M\left(A^{-1}\right)^\ast=\operatorname{diag}(t_1,t_2-t_1,\dots,t_n-t_{n-1}).
\end{align*}
which implies that (B1) and (B2) holds. \qedhere
\end{enumerate}
\end{proof}
\end{thm}
%Kasten

\begin{prop}
Let $\textbf{B}^1$,\dots,$\textbf{B}^d$ be $1$-dimensional Brownian motions and
let the $(\textbf{B}^i)_{i=1,\dots,d}$ be independent (as stochastic processes).
Then $\left(B_t^1,\dots,B_t^d\right)_{t\geq 0}$ is a $d$-dimensional Brownian motion.
Conversely, the coordinate processes $\left(B_t^i\right)_{t \geq 0}$ of a $d$-dimensional Brownian motion are independent $1$-dimensional Brownian motions.
\begin{proof}
This is remains as an exercise or can be found in Section 2.3 of Schilling/Partzsch.
\end{proof}
\end{prop}

\begin{prop}
Let $\textbf{B}$ be a $1$-dimensional Brownian motion. Then its fdd are given by
\begin{align}
p_{t_1,\dots ,t_n}(A_1\times \dots \times A_n) = \Prob\left(B_{t_1\in A_1,\dots,B_{t_n}\in A_n}  \right) \nonumber \\
=\frac{1}{(2\pi)^{\frac{n}{2}}}\frac{1}{\left[\prod_{j=1}^n(t_j-t_{j-1})\right]} \int_{A_1\times \dots \times A_n} \exp\left(-\frac{1}{2}\sum_{j=1}^n \frac{(x_j-x_{j-1})^2}{t_j-t_{j-1}}\right)~\mathrm{d}x
\end{align}
%TODO gleichung iwie schoen machen
for all $0=t_0<t_1<\dots < t_n$, $A_1, \dots , A_n \in  \Borel(\R), n\in \N$ with $x_0=0$ and $x=(x_1,\dots , x_n)$.
\begin{proof}
Referring to Thm 1.20 this remains as an exercise.
\end{proof}
\end{prop}

\begin{prop}
The family of fdd given in the previous proposition is consistent in the sense of (1.17),(C1) and (C2).
%TODO label hinmachen
\end{prop}

So, Brownian motions have a chance to exist. We now that it does. Nevertheless, this will take a while.

\begin{defi}
Let $(E,\mathcal{E})$ be a measurable space and $T$ a set.
\begin{enumerate}[label=\roman*)]
\item The map 
\begin{align*}
pi_t&\colon E^T\to E\\
(e_s)_{s\in T} &\mapsto e_t
\end{align*}
is called \emph{coordinate projection} to the $t$-th coordinate.
When we identify $E^T$ with $\{f \colon T \to E\}$ then $\pi_t(e)=e_t$ is the point evaluation of the function $e$ at point $t$.
\item The $\sigma$-algebra $\mathcal{E}^{\otimes T}$ is the smalles $\sigma$-algebra on $E^T$ so that all maps $\pi_t$ are $\mathcal{E}^{\otimes T}$-$\mathcal{E}$-measurable.
\item The measurable space $(E^T,\mathcal{E}^{\otimes T})$ is the \emph{canonical space} for $E$ valued stochastic processes with index set $T$.
\item If $\Omega_0\subset E^T$ is any subset (not necessarily measurable), the $\sigma$-algebra
\begin{align*}
\mathcal{E}^{\otimes T}\cap \Omega_0 \coloneqq \{A \cap \Omega_0 \colon A \in \mathcal{E}^{\otimes T}\}
\end{align*}
is called the \emph{trace} of $\mathcal{E}^{\otimes T}$ on $\Omega_0$.
The measurable space $(\Omega_0,\mathcal{E}^{\otimes T}\cap \Omega_0)$
is the canonical space for $E$-valued process with sample paths in $\Omega_0$.
\end{enumerate}
\end{defi}

\begin{bsp}
$E=\Rd$, $T=\R_0^+$ and $\Omega=E^T=\{\omega \colon \R_0^+ \to \Rd\}$,
$\pi_t(\omega)=\omega(t)$, $\Omega$=space of all \glqq paths\grqq\, $t \to \omega(t)$.
Write $X_t(\omega)=\pi_t(\omega)=\omega(t)$.
We consider $\Omega \cap C_0(\Rd)=\{\omega \in C(\R_0^+,\Rd), \omega(0)=0\}$ and $\mathcal{F}=\Borel(\Rd)^{\otimes \R_0^+} \cap C_0(\Rd)$.
Then $(C_0(\Rd),\mathcal{F})$ is the canonical measurable space for a stochastic process with continuous paths.
\end{bsp}
%TODO improve language here, find right C for continouous fcts.

\begin{bem}
The metric of \emph{local uniform convergence} on $C_0(\Rd)$ is given by 
\begin{align*}
\rho &\colon C_0\times C_0 \to \R_0^+\\
(f,g) &\mapsto \sum_{n=1}^\infty \min\{1,\sup_{0\leq t \leq n}\abs{f(t)-g(t)}\} 2^{-n}. 
\end{align*}
The Borel-$\sigma$-algebra $\Borel(C_0)$ on $C_0$ is the smallest $\sigma$-algebra on $C_0$ such that all $\rho$-open sets are measurable.
We have $\Borel(C_0)=\Borel(\Rd)^{\otimes \R_0^+}\cap C_0$.
\begin{proof}
This remains as an exercise.
\end{proof}
\end{bem}

\begin{lem}
Let $(E,\mathcal{E})$ be a measurable space, $T$ a set and $A \subseteq E^T$. Then $A \in \mathcal{E}^{\otimes T}$ if and only if there exists $I\subseteq T$ countable with $A \in \{\pi_t\colon t\in I\}$.
\begin{proof}
This remains as an exercise (on some exercise sheet).
\end{proof}
\end{lem}
%important
Recall the following:
\begin{thm}[Thm 3.29 from Probability Theory, Winter Term 17/18]
Let $(\Omega,\mathcal{F})$ be a Borel space and $\Prob$ a probability measure on $(\Omega,\mathcal{F})$.
Then for each $\sigma$-algebra $\mathcal{G} \subseteq \mathcal{F}$, a map
$\mu \colon \Omega \times \mathcal{F} \to [0,1]$ with the properties
\begin{enumerate}[label=(\roman*)]
\item $\mu(\cdot,A)$ is $\mathcal{F}$-measurable for all $A \in \mathcal{F}$
\item $\mu(\omega,\cdot)$ is a probability measure for all $\omega \in \Omega$
\item Moreover, $\mu(\omega,\cdot)$ is a conditional probability of $A$ given $\mathcal{G}$, \ie $\mu(\omega,A)=\Prob(A\mid \mathcal{G})(\omega)$ for $\Prob$-almost all $\omega \in \Omega$.
\end{enumerate}
exists. $\mu$ is called \emph{regular conditional probability}.
\begin{proof}
Will be uploaded in the notes (to be done later).
\end{proof}
\end{thm}

\begin{lem}
Let $(\Omega,\mathcal{F})$ be a Borel space, $\mu \colon \Omega \times \mathcal{F} \to [0,1]$ with properties (1.28)(i),(ii) [called a \emph{probability kernel}].
Then there exists a $\mathcal{U}[0,1]$-RV $Y$ and an $\mathcal{F} \otimes \Borel([0,1])$-measurable function
$f \colon \Omega \times [0,1] \to \Omega$ with
\begin{align*}
\mu(\omega,A)=\Prob(f(\omega,Y)\in A)= \int_0^1 \mathds{1}_{\{f(\omega,\cdot) \in A\}}(u)~\mathrm{d}u
\end{align*}
for all $\omega \in \Omega$, $A \in \mathcal{F}$.
\begin{proof}
This remains as an exercise or can be found in Kallenberg [Foundations of Modern Probability, 3.22].
\end{proof}
\end{lem}

\begin{thm}
Let $(E,\mathcal{E})$ be Borel.
For each $n \in \N$ let $\Prob_n$ be a probability measure on $(E^n,\mathcal{E}^{\otimes n})$ and assume \emph{consistency}, \ie for all
$n \in \N$ and all $A \in \mathcal{E}^{\otimes n}$ it holds that
\begin{align*}
\Prob_{n+1}(A\times E)=\Prob_n(A).
\end{align*}
Let $(\Omega,\mathcal{F},\Prob)=\left([0,1]^\N,\Borel([0,1])^{\otimes \N},\mathcal{U}([0,1])^{\otimes \N}\right)$.
Then there exist random variables $X_i \colon \Omega \to E$, $i \in \N$ such that for all $n \in \N$ and all $A \in \mathcal{E}^{\otimes n}$ it holds that
\begin{align*}
\Prob_n(A)=\Prob((X_1,\dots,X_n)\in A).
\end{align*}
\end{thm}

\begin{proof}
1. Fix $n \in \N$.
Then $(E^{n+1},\mathcal{E}^{\otimes n+1})$ is Borel as a product of Borel spaces (exercise!).
We set $\mathcal{G}_n \coloneqq \sigma (\{A_1\times \dots \times A_n \times E  \colon A_i \in \mathcal{E}\})$ and $\mu_n \colon E^{n+1}\times \mathcal{E}^{\otimes n+1}\to  [0,1]$ as in (1.28),
%TODO check if that is the right number
\ie $\mu_n(\textbf{x},A)=\Prob_{n+1}(A \mid \mathcal{G}_n)(x)$ almost surely with respect to $\P_{n+1}$ for all $A \in \mathcal{E}^{\otimes n+1}$.
Since $\textbf{x} \mapsto \mu_n(\textbf{x},A)$ is measurable with respect to $\mathcal{G}_n$, it depends only on $x_1,\dots,x_n$ and not on $x_{n+1}$.
Write
\begin{align*}
\tilde{\mu_n}((x_1,\dots,x_n),A)=\mu_n(\textbf{x},A).
\end{align*}
Note that $\mu_0(\textbf{x},A)=\Prob_1(A)$ does not depend on $x_1$.
%2. Write $\textbf{\omega}\in \Omega$ as $\textbf{\omega}=(\omega_1,\omega_2,\dots)$ and let $Y_i(\textbf{\omega})\coloneqq \omega_i$.
%Then the random variables $(Y_i)$ are iid, $\mathcal{U}[0,1]$-distributed random variables.
3. By (1.29), there exist functions $f_n \colon E^n \times [0,1]\to E$ with $\mu_n(\textbf{x},A)=\Prob(f(x_1,\dots,x_n,Y_{n+1})\in A)$.
%TODO check number
In particular
\begin{align*}
\mu_0(\textbf{x},A)=\tilde{\mu_0}(A)=\Prob(f_0(Y_1)\in A).
\end{align*}
Now, we will proceed by induction.
Put $X_1=f(Y_1)$.
Assume that $(X_1,\dots,X_n)$ have been constructed.
We set $X_{n+1}(\omega)\coloneqq f_n((X_1,(\omega),\dots,X_n(\omega),Y_{n+1}(\omega))$.
Since $\mu_{n}(\textbf{x},A)=\tilde{\mu}_{n}(x_1,\dots,x_n),A)=\Prob(f_{n+1}(x_1,\dots,x_n,Y_{n+1})\in A)$,
we find that for all $A_1,\dots,A_{n+1}\in \mathcal{E}$, it holds that
\begin{align*}
\Prob(X_i \in A_i \forall i \leq n+1)&=\E{\Prob(X_{n+1}\in A_{n+1}\mid \mathcal{G}_n)\prod_{i=1}^n \mathds{1}_{\{x_i \in A_i\}}}\\
&=\E{\Prob(f_{n}((X_1,\dots,X_n),Y_{n+1})\in A_{n+1}\mid \mathcal{G}_n)\prod_{i=1}^n \mathds{1}_{\{X_i \in A_i\}}}\eqqcolon (\ast).
\end{align*}
%TODO see how he uses \ast and maybe simply give the equation a number
Since $Y_{n+1} \amalg (X_1,\dots,X_n)$, Proposition 3.23 from Probability Theory implies that
\begin{align*}
\Prob(f_n((X_1,\dots,X_n),Y_{n+1})\in A_{n+1}\mid \mathcal{G}_n)(\omega)&=\Prob(f_n((X_1(\omega),\dots,X_n(\omega)),Y_{n+1})\in A_{n+1})\\
&=\tilde{\mu_n}((X_1(\omega),\dots,X_n(\omega)),A_{n+1})
\end{align*}
holds $\Prob$-almost surely.
%TODO overfull box 
So using the image measure we have
\begin{align*}
(\ast)&=\E{(\tilde{\mu_n}((X_1,\dots,X_n),A_{n+1})\prod_{i=1}^n \mathds{1}_{\{X_i \in A_i\}}}\\
&=\int \tilde{\mu_n}((x_1,\dots,x_n),A_{n+1})\prod_{i=1}^n \mathds{1}_{\{x_i \in A_i\}}\Prob_n(\mathrm{d}\textbf{x})\\
&=\int \tilde{\mu_n}((x_1,\dots,x_n),A_{n+1})\prod_{i=1}^n \mathds{1}_{\{x_i \in A_i\}} \Prob_{n+1}(\mathrm{d}\textbf{x})\\
&=\E{\Prob_{n+1}(A_{n+1}\mid \mathcal{G}_n)\prod_{i=1}^n \mathds{1}_{A_i}}\\
&=\E{\prod_{i=1}^{n+1} \mathds{1}_{A_i}}=\Prob_{n+1}(A_1,\times \dots \times A_{n+1}).
\end{align*}
%TODO underbrace im dritten term \mu_1(\textbf{x},A))
Then the claim follows by induction.
\end{proof}
%TODO find a good way for enumerating this proof

\begin{thm}[Kolmogorov 1932]
Let $(E,\mathcal{E})$ be Borel, $T$ a set. Let $\{p_{t_1,\dots,t_n} \colon t_1,\dots,t_n \in T,n \in \N\}$ be a family
of probability measures, that are consistent, \ie fulfill (C1),(C2) of (1.17).
%TODO rearrange sentence such that there is no overfull box anymore
Then there exists a probability measure $\Prob$ on $(E^T,\mathcal{E}^{\otimes T})$ such that
\begin{align*}
p_{t_1,\dots,t_n}(A)=\Prob((\Pi_{t_1},\dots,\Pi_{t_n})\in A)
\end{align*}
holds for all $A\in \mathcal{E}^{\otimes T}$ and all $t_1,\dots,t_n, n\in \N$.
\end{thm}

\begin{proof}
Let $A \in \mathcal{E}^{\otimes T}$.
By Lemma 1.27, there exists a countable subset $I \subset T$ with $A \in \sigma(\Pi_t \colon t \in I)$.
We write $A)B \times E^{T \setminus I}$ for some $B \in \mathcal{E}^{\otimes I}$.
By the previous theorem, there exists a unique probability measure $\Prob_I$ on $\mathcal{E}^{\otimes I}$ with
\begin{align*}
p_{t_1,\dots,t_n}(A_1\times \dots \times A_n)=\Prob_I(\Pi_{t_i} \in A_i \forall i \leq n)
\end{align*}
for all $A_1,\dots,A_n,t_1,\dots,t_n$ and $n \in \N$.
\end{proof}
Define $\Prob(A)=\Prob_I(B)$ if $A=B \times E^{T\setminus I}$ for some countable $I\subseteq I$ and some $B \in \mathcal{E}^{\otimes I}$.
By consistency, $\Prob$ is well-defined and finitely additive.
For $\sigma$-additivity, let $(A_n)_{n\in \N}\subseteq \mathcal{E}^{\otimes T}$ be disjoint.
Then for each $n$ there exists a countable set $I_n \subseteq T$ and $B_n \in \mathcal{E}^{\otimes T_n}$ with $A_n=B_n \times E^{T\setminus I_n}$.
Set $I=\bigcup_{n\in \N}I_n$ the we have $A_n=\tilde{B}_n \times E^{T \setminus I}$ for all $n$ and the $\tilde{B}_n$ are in $\mathcal{E}^{\otimes I}$ and are disjoint.
Thus,
\begin{align*}
\Prob\left(\cup_{n \in \N}A_n\right)=\Prob_I \left( \cup_{n\in \N} \tilde{B}_n\right)=\sum_{n=1}^\infty \Prob_I(\tilde{B}_n)=\sum_{i=1}^n \Prob(A_1),
\end{align*}
and we are done.

\begin{cor}
An $\Rd$-valued stochastic process fulfilling (B0)-(B3) from (1.4) exists.
Explicitly, there exists a unique probability measure $\mathcal{W}$
%TODO math...?
on $\left(\left(\Rd\right)^{\R_0^+},\left(\Borel(\Rd)\right)^{\otimes \R_0^+}\right)$ such that the random variables
\begin{align*}
B_t \colon \Omega &\to \Rd\\
\omega &\mapsto B_t(\omega)\coloneqq \omega(t)
\end{align*}
fulfill (B0)-(B3). $\mathcal{W}$ is called \emph{pre-Wienermeasure}
\end{cor}
%TODO and also named definition
It remains to see (B4).
Note that the statement $\mathcal{W}(C(\R^+,\Rd))=1$ makes no sense, as $C(\R^+,\Rd)\not \in \mathcal{F}$. We need to be more careful.

\begin{defi}
Let $D\subseteq \R_0^+$ be open and $\alpha>0$.
A function $f \colon \R_0^+\to \Rd$ is called \emph{$\alpha$-Hoelder continuous on $D$} if the Hoelder-seminorm is finite.
$f$ is locally $\alpha$-HC on $D$ if it is Hoelder-continuous on $D$ if it is Hoelder-continuous on each $D \cap [0,n]$.
We write $f \in C^\alpha(D)$ or $f \in C_{loc}^\alpha(D)$
\end{defi}
%TODO mathoperator for loc and continuous functions

\begin{bem}
\begin{enumerate}[label=\alph*)]
\item If $D$ has no cluster points, any function on $D$ is an element of $C_{loc}^\alpha$.
\item Usually, $D$ is a dense subset of some interval $[a,b]\subseteq \R_0^+$.
\item In the case of b), $f \in C^\alpha(D)$ can be uniquely extended to $[a,b]$ by the usual extension.
Uniqueness follows from the continuity.
\item If $D$ is dense and $f \in C_{loc}^\alpha(D)$ for some $\alpha>1$, then $f$ is constant.
\item $f \in C_{loc}^1(\R)$ if and only if $f$ is locally Lipschitz.
\item $\norm{\cdot}_{D,\alpha}$ is only a seminorm, as it does not detect constants.
%unterscheiden constants from 0
\item The map $\norm{\cdot}_{D,\alpha}$ is $\Borel(\Rd)^{\otimes \R_0^+}$-$\Borel([0,\infty])$-measurable if $D$ is countable.
\end{enumerate}
\end{bem}

\begin{thm}
Let $(X_t)_{t\in [0,T]}$ be an $\Rd$-valued stochastic process.
Assume that there exist $q\geq 2$, $\beta >\frac{1}{q}$, $C<\infty$ such that
\begin{align}\label{eq:condition136}
\E{\abs{X_{s,t}}^q} \leq C \abs{t-s}^{\beta q}
\end{align}
%TODO better label
holds for all $s,t \in [0,T]$ with $\abs{t-s}<\frac{1}{2}$.
Let $D\coloneqq\{k \cdot 2^{-n} \colon k,n \in \N\} \cap [0,T]$ (dyadic rationals).
Then
\begin{align*}
\E{\norm{X}_{D,\alpha}^q}<\infty
\end{align*}
holds for all $\alpha \in [0,\beta-\frac{1}{q})$.
\end{thm}
%TODO alpha=0?
\begin{proof}
Let $D_n \coloneqq \{k \cdot 2^{-n}\} \cap [0,T]$. Then we have $D=\bigcup_{n\in \N}D_n$.
%remark: dn finite
We define
\begin{align*}
K_n(\omega) \coloneqq \max\{\abs{X_{t,t+2^{-n}}(\omega)}\colon t \in D_n\}.
\end{align*}
Then by \eqref{eq:condition136} it holds that
\begin{align*}
\E{K_n^q} &\leq \E{\sum_{t \in D_n} \abs{X_{t,t+2^{-n}}}^q}\\
&\leq \abs{D_n}C (2^{-n})^{\beta q}\leq T 2^n C (2^{-n})^{\beta q}\\
&=C T (2^{-n})^{\beta q-1}.
\end{align*}
%gleichung stern
Now let $s<t$, $s,t\in D$. If $\abs{t-s}>\frac{1}{2}$ we pick $t_1,\dots,t_m\in D$ with $t_0=s$, $t_m=t$ and $\abs{t_i-t_{i-1}}<\frac{1}{2}$ for all $i$.
Since $\abs{X_t-X_s} \leq \sum_{i=1}^m \abs{X_{t_i}-X_{t_{i-1}}}$, we find
\begin{align*}
\frac{\abs{X_t-X_s}}{\abs{t-s}^\alpha} &\leq \sum_{i=1}^n \frac{\abs{X_{t_i}-X_{t_{i-1}}}}{\abs{t_i-t_{i-1}}^\alpha}\\
&\leq m \sup\{\frac{\abs{X_t-X_s}}{\abs{t-s}} \colon s,t \in D, \abs{t-s}<\frac{1}{2}\}.
\end{align*}
So it suffices to consider $\abs{t-s}\leq \frac{1}{2}$.
Then there exists $j \in \N$ with $2^j<t-s \leq 2^{-j-1}$ and $N \in \N$ so that $s,t \in D_N$.
We define $A_j\colon D_j \cap (s,t)$, then $\abs{D_j} \in \{1,2\}$.
Pick $t_j^- \coloneqq \min A_j$ as well as $t_j^+ \coloneqq \max A_j$ and set
$A_{j+1}^- \cap [s,t_j^-)$ then $\abs{A_{j+1}^-}\in \{0,1\}$.
Pick $t_{j+1}^-\coloneqq \min\{t_j^{-1}, \inf A_{j+1}^-\}$.
Analogously for $t_{j+1}^+$ via $A_{j+1}^+$ with $\min$ and $\inf$ replaced by $\max$ and $\sup$.
Inductively, $A_{j+l}^- \coloneqq D_{j+l}\cap [s,t_{j+l-1}^-)$, $t_{j+l}^-\coloneqq \min\{t_{j+l-1},\inf A_{j+l}^-$, $A_{j+l}^+\coloneqq D_{j+l} \cap (t_{j+l-1}^+,t]$, $t_{j+l}^+ \coloneqq \max\{t_{j+l-1}^+,\sup A_{j+l}^+\}$.
%TODO in align machen
This will stop when $j+l=N$ with $t_N^-=s,t_N^+=t$.
Now,
\begin{align*}
\abs{X_{s,t}} &\leq \sum_{l=0}^{N-j} \abs{X_{t_{j+l}^-}(\omega)-X_{t_{j+l-1}^-}(\omega)}+ \sum_{l=0}^{N-j} \abs{X_{t_{j+l}^+}(\omega)-X_{t_{j+l-1}^+}(\omega)}
\end{align*}
%der term bekommt \ast \ast
Each term is either equal to $0$ or the difference of some
$\abs{X_{t_{j+l}^\pm}-X_{t_{j+l}^\pm+2^{-(j+l)}}}$.
So, $(\ast \ast)\leq 2 \sum_{l=0}^{N-j} K{j+l}(\omega)\leq 2 \sum_{l=j}^\infty K_l(\omega)$.
Since we assumed $\abs{t-s}>2^{-j}$, we get
\begin{align*}
\frac{\abs{X_{s,t}(\omega)}}{\abs{t-s}^\alpha} &\leq 2^{j\alpha} \cdot 2 \sum_{l=j}^\infty K_l(\omega)\\
&\leq 2 \sum_{l=0}^\infty 2^{l\alpha} K_l(\omega).
\end{align*}
The right hand side above does not depend on $s$ and $t$.
Thus,
\begin{align*}
\E{\norm{X}_{D,\alpha}^q}^{\frac{1}{q}} &\leq \E{(2T)^q (2 \sum_{l=0}^\infty 2^{l\alpha} K_l)^q}^\frac{1}{q}\\
&=4T \abs{\sum_{l=0}^\infty 2^{l\alpha} K_l}_{\mathit{L}^q} \leq 4T \sum_{l=0}^\infty 2^{l\alpha} \abs{K_l}_{\mathit{L}^q}\\
&\leq 2CT \sum_{l=0}^\infty 2^{l\alpha} 2^{-l \frac{\beta q -1}{q}}
\end{align*}
where we used $(\ast)$ in the last step.
%mathit richtig?
For $\alpha <\beta -\frac{1}{q}$ this sum is finite.
\end{proof}

\begin{cor}
Let $(X_t)_{t\in \R_0^+}$ be an $\Rd$-valued stochastic process on $(\Omega,\mathcal{F},\Prob)$ which fulfills \eqref{eq:condition136} for all $T$.
\begin{enumerate}[label=\alph*)]
\item(Kolmogorov-Chentsov-Than) There exists a stochastic process $(\tilde{X}_t)_{t \in \R_0^+}$ on $(\Omega,\mathcal{F},\Prob)$ such that
\begin{enumerate}[label=\roman*)]
\item for all $\omega \in \Omega$ the map $t \mapsto \tilde{X}_t(\omega)$ is continuous, we say that $\tilde{X}$ has continuous paths.
\item for all $ \in \R_0^+$ we have $\Prob(X_t=\tilde{X}_t$, this is to say $\tilde{X}$ is a \emph{version} of $X$.
\end{enumerate}
%namen richtig?!
\item On the canonical space $(C(\R_0^+,\Rd),\mathcal{F}_{can})$
%TODO can im textmodus
there exists a probability measure $\Prob_0$ so that with
\begin{align*}
Y_t(\hat{\omega)}=\hat{\omega}(t)~~\forall \hat{\omega} \in C(R_0^+,\Rd)
\end{align*}
the processes $(X_t)$ and $(Y_t)$ have the same fdd.
\end{enumerate}
\end{cor}

\begin{proof}
\begin{enumerate}[label=\alph*)]
By Thm 1.36 we have $\E{\norm{X}_{D_T,\alpha}^q}<\infty$ for some $\alpha>0,q\geq 2$ and all $D_T=D_{\R_0^+}\cap [0,T]$.
Hence $\Prob(\norm{X}_{D_T,\alpha}<\infty)=1$ for all $T$ and therefore $\Prob(X\in C_{loc}^\alpha (D))=1$.
The set $\{\omega \in \Omega \colon X(w) \in C_{loc}^\alpha(D))\eqqcolon \Omega_0$ depends on countably many indices and is thus measurable.
Define
\begin{align*}
\tilde{X}_t(\omega)\coloneqq \begin{cases}
\lim_{t_n \to t} X_{t_n}(\omega),~~&\text{for } t_n \in D, w \omega \in \Omega_0,\\
0, &\text{otherwise }.
\end{cases}
\end{align*}
$\tilde{X}_t(\omega)$ is independent of the approximating sequence and $t \mapsto \tilde{X}_t(\omega)$ is continuous for all $\omega \in \Omega$.
Hence (i) holds. For (ii), let
\begin{align*}
Z_t(\omega)\coloneqq X_t(\omega) -\tilde{X}_t(\omega)\\
Z_{t_n} \coloneqq X_t(\omega)-\tilde{X}_{t_n}(\omega)
\end{align*}
for $t_n \in D$, $t_n \to t$.
Then $Z_{t_n}(\omega)\to Z_t(\omega)$ on $\Omega_0$, \ie almost surely, and by \eqref{eq:condition136}
\begin{align*}
\Prob(\abs{Z_{t_n}}>\varepsilon)&\leq \frac{1}{\varepsilon^q}\E{\abs{X_{t_n}-X_t}^q}\\
&\leq C \frac{1}{\varepsilon^q}\abs{t_n-t}^{\beta q} \to 0,
\end{align*}
$Z_{t_n}\to 0$ in probability and hence $Z_t=0$ almost surely.
\item The map 
\begin{align*}
F \colon \Omega &\to \Omega_0\\
\omega &\mapsto (\tilde{X}_t(\omega))_{t\in \R_0^+}
\end{align*}
is measurable.
Then $\Prob_0=\Prob \circ F^{-1}$. \qedhere
\end{enumerate}
\end{proof}

\begin{thm}
\begin{enumerate}[label=\alph*)]
\item There exists a stochastic process satisfying B0-B4, \ie a Brownian motion.
\item There exists a probability measure $\mathcal{W}$ on $(C_0,\mathcal{F})$ such that under $\mathcal{W}$ the projection $\omega \mapsto \pi_t(\omega)=B_t(\omega)=\omega(t)$ form a Brownian motion.
The space $(C_0,\mathcal{F},\mathcal{W})$ is called \emph{Wiener space}.
\end{enumerate}
\end{thm}

\begin{proof}
\begin{enumerate}[label=(\roman*)]
\item By 1.22 we can restrict to 1-d BM.
By Cor 1.33 a process $\tilde{B}_t)$ with B0-B3 exists. By B3
\end{enumerate}
\end{proof}