\chapter{Brownian Motion}
\begin{defi}
Let $(E,\Sigma)$ be a measurable space and $T$ a set.
A collection of $(E,\Sigma)$-valued random variables (RVs) $(X_t)_{t\in T}$ is called \emph{$E$-valued stochastic process (SP) with index set $T$}
\end{defi}
%TODO Kasten

\begin{bsp}
\begin{enumerate}[label=(\alph*)]
\item $(E,\Sigma)=(\Rd,\mathcal{B}(\Rd))$ and $T=\R_{\geq 0}$ then a SP is called \emph{$\Rd$-valued, continuous-time SP},
\item $(E,\Sigma)=(\{-1,1\},\mathcal{P}(\{-1,1\}))$ and $T=\Z^d$, \glqq spin system\grqq\, ,
\item If $E$ is countable and $T=\N_0$ we speak of an \emph{time discrete SP}.
\end{enumerate}
\end{bsp}

\begin{bem}
From a dynamical point of view, $(X_t)$ is a $t$-dependent quantity that changes with time, this is suitable for the first and third example.
From a global point of view, a SP is one RV with values in the space $\Omega=E^T=\{f\colon T\to E\}$.
In the first example this means to consider the whole \emph{path} $(X_t)_{t\geq 0}$ as one object.
In the second example, one \glqq spin configuration\grqq $\in \{-1,1\}^{\Z^d}$ is an element of $\Omega$.
Questions: $\sigma$-algebra? Existence?
\end{bem}

\begin{defi}
Let $\textbf{X}=(X_t)_{t\in T}$ be a SP where the state space $E$ is a group (e.g $E=\Rd$) and $T\subseteq \R$.
The set $(X_{s,t})_{s,t\in T}$ with $X_{s,t}\coloneqq X_t-X_s$ is called the \emph{increment process of $\textbf{X}$} or \emph{set of increments}.
A SP has \emph{independent increments} if for all $n\in \N$ and all $s_1<t_1\leq s_2 <t_2\leq \dots \leq s_n<t_n$ with $s_i,t_i \in T$, the RVs $(X_{s_i,t_i})_{1\leq i \leq n}$ are independent.
A SP has \emph{stationary increments} if for all $n\in \N$ and all $s_1<t_1\leq s_2 <t_2\leq \dots \leq s_n<t_n, r$ with $s_i,t_i,r \in T$, we have
\begin{align*}
(X_{s_i,t_i})_{i=1,\dots,n}\sim  (X_{s_i+r,t_i+r})_{i=1,\dots,n},
\end{align*}
i.e. equal in distribution. 
\end{defi}
%TODO Kasten

\begin{defi}
A $\Rd$-valued SP $\textbf{B}=(B_t)_{t \in \R_{\geq 0}}$ is called \emph{Brownian Motion} (BM) if 
\begin{enumerate}
\item[(B0)] $B_0(\omega)=0$ for almost all $w\in \Omega$,
\item[(B1)] $\textbf{B}$ has independent increments,
\item[(B2)] $\textbf{B}$ has stationary increments,
\item[(B3)] $B_t-B_s\eqqcolon B_{s,t}\sim B_{t-s}\sim \mathcal{N}(0,(t-s)I_d)$
\item[(B4)] The map $t\mapsto B_t(\omega)$ is continuous for all $\omega \in \Omega$.
\end{enumerate}
\end{defi}
%TODO Kasten

\begin{bem}
Interpretation of (B0)-(B4) in view of $B_t=\int_0^t \xi_s~\mathrm{d}s$.
\begin{enumerate}
\item[(B0)] $\int_0^0\xi_s~\mathrm{d}s=0$,
\item[(B1)] $B_t-B_s=\int_s^t \xi_r~\mathrm{d}r$ and $\xi_r \amalg (\xi_s)_{s\not =r}~~\forall r$
\item[(B2)] distribution of $\xi_r$ does not depend on $r$,
\item[(B3)] Central limit theorem and Riemann approximation,
\item[(B4)] $t\to \int_0^t f_s~\mathrm{d}s$ is continuous for all \glqq sensible\grqq\, functions $f$, in particular for $f=\xi$.
\end{enumerate}
\end{bem}

\begin{defi}
The \emph{Gaussian measure} $\mathcal{N}(m,\sigma^2)$ with \emph{mean} $m$ and \emph{variance} $\sigma^2$ is the probability measure on
$(\R,\mathcal{B}(\R))$ with Lebesgue density
\begin{align*}
g_{m\sigma^2}(x)=\frac{1}{\sqrt{2\pi \sigma^2}} \exp\left(-\frac{1}{2\sigma^2}(x-m)^2\right).
\end{align*}
\end{defi}

\begin{prop}
Let $X\sim \mathcal{N}(m,\sigma^2)$. Then
\begin{enumerate}[label=(\alph*)]
\item $\mathds{E}(X)=m, \mathds{V}(X)=\sigma^2$.
\item We have the \glqq Gaussian tail estimate\grqq
\begin{align*}
\frac{1}{\sqrt{2\pi}} \frac{C}{C^2+1}\mathrm{e}^{-\frac{C^2}{2}}\leq \mathds{P}(X-m\geq C \sigma)\leq \frac{1}{\sqrt{2\pi}} \frac{1}{C}\mathrm{e}^{-\frac{C^2}{2}},
\end{align*}
for all $C>0,\sigma>0$.
\item For $(m_k)_{k\in \N}\subseteq \R$, $m\in \R$, $(\sigma_k)_{k\in \N}\subseteq \R_{\geq 0}$, $\sigma\in \R_{\geq 0}$ we have:
$m_k\to m$ and $\sigma_k\to \sigma$ if and only if $\mathcal{N}(m_k,\sigma_k^2)\overset{\text{d}}{\to} \mathcal{N}(m,\sigma^2)$.
\end{enumerate}
\end{prop}

\begin{defi}
A $\Rd$-valued RV $X$ is called \emph{$d$-dimensional Gaussian} if for all linear $L\colon \Rd\to \R$ there exist $m,\sigma^2$ with $LX\sim \mathcal{N}(m,\sigma^2)$.
Explicitly: if $X=(X^1,\dots,X^d)$ this means that for all $a_1,\dots,a_d \in \R$ there exist $m,\sigma^2$ such that $\sum_{i=1}^d a_i X^i \sim \mathcal{N}(m,\sigma^2)$.
\end{defi}