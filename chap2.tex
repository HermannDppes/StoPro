\def\BM{Brownian Motion}
\chapter{Properties of \BM}

\BM s have a lot of the general properties an SP might have.
Hence, we can do a lot with \BM s we can not do in a more general setting.

\section{Invariance properties}
The statements are far more interesting than the proofs,
which are too easy to help in understanding the properties.
We can do all kinds of funny things to \BM and obtain \BM s again.

When we just transform the values of a \BM,
we can get new \BM s quite easily.

\begin{prop}[Orthogonal invariance]
	Let \(\B\) be a \def\BMd{\BM\textsuperscript{d}}\BMd
	and \(U\) d by d orthogonal matrix
	then \((U\B_t)_{t\in T}\) is a \BMd.
	In particular, \(-\B = (-\B_t)_{t\in T}\) is a \BMd.
\end{prop}
% Kasten

\begin{proof}
	With \(UU^* = \Id\) we have …
\end{proof}

Instead of transforming the values,
we can also shift the process in time.
This is an important property which to understand will be quite usefull.

\begin{prop}[Time shift invariance]
\end{prop}
% Kasten
% TODO: Graphic depiction of time shift (lots o blood)
%\begin{tikzpicture}
%	\newcommand*\coordinates[2][]{%
%		\begin{scope}}
%\end{tikzpicture}

The next property is very fundamental
and central to our understanding of \BM s.
The path a \BM takes is entirely independent of the path taken so far
except for the value the \BM now starts at.
If we forget where we are
we essentially delete all memory of the past.
For all relevant purposes,
the value we take at the current point in time
is all we know about what has happened so far.

\begin{prop}[Memoryless property, elementary Markov property]
	\(\timeshift\B a\)
\end{prop}
% Kasten

\begin{proof}
	Idea: fdd and intersection stable generator of the \(\sigma\)-algebras
	Transform the fdd with \[A = \lowtriangones\]
	to get another int stable generator.
	Do this for both processes.
	\[W_{t_i}-W_{t_{i-1}} = B_{t_i+a}-B_{t_{i-1}+a}\]
	Use (B1).
	P-Theo: Indep of int-stable generator means indep of sigma-alg.
\end{proof}

A diffusive rescaling is a rescaling of a process
where we rescale the time,
\eg let it run twice as fast,
where we also transform the values to offset the rescaling.
If we were to double the speed of a linear function
and then were to divide the values by \(2\)
we would obtain the original function again.
In case of \BM we obviously will not be able to recover the exact function
but we can get a process of identical distribution.

\begin{prop}[Invariance under diffusive rescaling]
\end{prop}
% Kasten

\begin{proof} Exercise\end{proof}

Imagine ink in water.
If we let time run 100 times as fast
the ink will seem to disperse at (merely) ten times the speed.

The function recovered exactly by this rescaling is the square root.
This means that a \BM looks “locally like a square root except it doesn't”.
\BM very rough and does not look like sqrt at all.
(It only looks like sqrt on small and large scales. –.–)
Sqrt leaves \(0\) very fast but gets slower over time.

If we stop a \BM at some point in time
and let it play backwards
we obtain a \BM again.

\begin{prop}[Time reversal symmetry]
\end{prop}
% Kasten
\begin{proof} Exercise\end{proof}

Another way in which \BM s look like the sqrt:
\[\sqrt t = t\sqrt{\frac1t}\]

relates behaviour at zero to behaviour at infinity
How a bm looks at infinity how it looks at zero
a very strong symmetry

The involution is not a \BM on the original probability space
since there will be paths that are not continuous in zero.
However, nearly all paths are continuous
and we can simply throw away the remaining points.

\begin{prop}[Time involution invariance]
\end{prop}
% Kasten
\begin{proof}
	The hard part of the proof is continuity at zero

	Something that is often very useful when dealing with SP:
	Write out definition of the limit in quantors over countable sets.
	Intersection with \(\Q\) is sufficient
	since we have continuity outside of zero % FIXME Really? Needs details

	\(\mathcal F \cap \Omega_0 \coloneqq \{F \cap \Omega_0 : F \in \mathcal F\}\)
\end{proof}

We now change our perspective.
So far, we took values and time and did things to that.
We took a function from R to R and transformed it in some way.
All of these can be written as operators on function spaces
to obtain measure-preserving maps.

\begin{bem}[Invariances of Wiener measure]
\end{bem}

Think of \(C_0\) like a big \(\R^n\).
Many maps will not preserve the measure but some do.

\section{Martingale properties of \BM}

In probability theory often only martingales on discrete sets are defined
but it works just as well on continuous sets.

\begin{defi}
	The third condition is the one that makes a martingale a martingale

	sub/super harmonic functions

	\def\bup{\sffamily b\hspace*{-7.6pt}\raisebox{3pt}{\(\uparrow\)}}
	\def\pdown{\sffamily p\hspace*{-7.6pt}\raisebox{-3pt}{\(\downarrow\)}}
	\def\submart{\textsf{su\rlap{\bup}\phantom{b}-martingale}}
	\def\supmart{\textsf{su\rlap{\pdown}\phantom{p}-martingale}}
	\submart\ \supmart
\end{defi}
% Kasten
