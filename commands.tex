%%% Commands %%%%%%%%%%%%%%%%%%%%%%%%%%%%%%%%%%%%%%%%%%%%%%%%%%%%%%%%%%%%%%%%%%%

% Common sets
\newcommand{\N}{\mathds{N}} % Natural numbers
\newcommand{\Z}{\mathds{Z}} % Integers
\newcommand{\Q}{\mathds{Q}} % Rationals
\newcommand{\R}{\mathds{R}} % Reals
% Overwriting a hyperref command
% We do not know what it would have done
\renewcommand{\C}{\mathds{C}} % Complex numbers

% Common constants
\newcommand{\e}{\mathrm{e}} % Euler's number, exp(1)
\renewcommand{\i}{\mathrm{i}} % Imaginary unit

% Common notations
\newcommand{\abs}[1]{\left\vert#1\right\vert} % Absolute value
\newcommand{\norm}[1]{\left\lVert#1\right\rVert} % Norm
\newcommand{\Po}{\mathcal{P}} % Powerset
\newcommand\Sym[1]{\mathrm{S}_{#1}} % Symmetric group
% Overriding superscript
\renewcommand*\sp[3][auto]{\argl{#1}\langle #2, #3\argr{#1}\rangle} % Scalar product

% Stochastical notations
\DeclareMathOperator{\Cov}{Cov} % Covariance matrix
\newcommand{\Expectedvalue}{\mathbb{E}} % The “expected value” E
\newcommand{\E}[2][auto]{\Expectedvalue\argl{#1}(#2\argr{#1})} % Expected value
\newcommand{\Variance}{\mathbb{V}} % The “variance” V
\newcommand{\V}[1]{\Variance\left(#1\right)} % Variance
\newcommand{\Normaldistribution}{\mathcal{N}} % The “normal distribution” N
\newcommand{\Ndist}[2]{\Normaldistribution\left(#1,#2\right)} % Normal dist.
\newcommand{\Prob}{\mathds{P}} % Probability measure
\newcommand{\Borel}{\mathcal{B}} % The Borel sets
\newcommand{\distto}{\overset{\text{d}}{\to}} % Convergence in distribution

% Environments for definitions, theorems etc.
% All of these are numbered together within each chapter
\swapnumbers % Numbers in front, e.g. “1.4 Theorem” instead of “Theorem 1.4”
\theoremstyle{default}
\newtheorem{defi}{Definition}[chapter]
\newtheorem{bsp}[ams@defi]{Example}
\newtheorem{ex}[ams@defi]{Exercise}

\newtheorem{thm}[ams@defi]{Theorem}
\newtheorem{prop}[ams@defi]{Proposition}
\newtheorem{lem}[ams@defi]{Lemma}
\newtheorem{cor}[ams@defi]{Corollary}

\newtheorem{bem}[ams@defi]{Remark}

% Texty abbreviations
\newcommand*\eg{e.g.\ }
\newcommand*\ie{i.e.\ }
\newcommand*\iid{i.i.d.\@\xspace}
\newcommand*\RV{random variable\xspace}
\newcommand*\RVs{random variables\xspace}
\newcommand*\SP{stochastic process\xspace}
\renewcommand*\textast{\ensuremath{\ast}}

% Mathy abbreviations
\newcommand*\X{\mathbf{X}}
\newcommand*\Y{\mathbf{Y}}
\newcommand*\B{\mathbf{B}}
\newcommand*\m{\mathbf{m}}
\newcommand*\x{\mathbf{x}}
\newcommand*\ba{\mathbf{a}}
\newcommand*\ud[1]{\mathop{\mathrm{d}{#1}}}
\newcommand{\Rd}{{\mathds{R}^d}} % The d-dimensional Euclidean space
\newcommand{\Id}{\mathrm{I}_d} % Unit matrix of \R^d

% Semantic commands
\newcommand*\defn[1]{\emph{#1}} % Definiendum

% Useful commands
\newcommand*\mailto[1]{\href{mailto:#1}{#1}}

% Technical helpers
\newcommand*\norml{}
\newcommand*\normr{}
\newcommand*\autol{\left}
\newcommand*\autor{\right}
\newcommand*\argl[1]{\csname #1l\endcsname}
\newcommand*\argr[1]{\csname #1r\endcsname}
