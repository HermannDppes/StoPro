\chapter{Motivation and Planned Topics}
Laws of nature are encoded by differential equations: ODEs, e.g. classical mechanics 
\begin{align*}
\partial_t^2 x(t)=-a(x(t))\partial_t x(t)+F(t,x(t))
\end{align*}
where $a$ is a friction coefficients and $F$ describes the force,
or PDEs e.g. electrodynamics or quantum mechanics.
The aim of this course is to describe ODEs with noise, for example with random force.
The simplest form of an ODE is
\begin{align}\label{eq:simpleform}
\partial_t X_t=b(t,X_t)+\sigma(t,X_t)\xi_t
\end{align}
where $b$ is a deterministic force (drift), $\sigma$ denotes the diffusion coefficient (tells how important the noise is) and as the most important part
$\xi_t$ is the white noise.
First question: what noise?
The \glqq natural\grqq\, approach is a random Fourier series:
\begin{align}
\xi_t^{(N)}\coloneqq \sum_{k=0}^N Y_k \cos(kt)+ \sum_{k=1}^N Z_k \sin(kt)~~~t\in [0,2\pi)
\end{align}
with $Y_k,Z_k \sim \mathcal{N}(0,1)$.
The limit $\lim\limits_{N\to \infty} \xi_t^{(N)}$ does \emph{not} exist and is called \glqq white noise\grqq.
Nevertheless, it exists as a tempered distributions.
A possible solution might be to integrate in time to improve convergence. For this we define
\begin{align}
B_t^{(N)}\coloneqq \int_0^t\xi_s^{(N)}~\mathrm{d}s=tY_0+\sum_{k=1}^N \frac{1}{k}(Y_k\sin(kt)-Z_k(\cos(kt)-1)).
\end{align}
Soon, we will see that this limit exists. It is called \glqq Brownian motion\grqq.
Assume in \eqref{eq:simpleform} that $\sigma(t,y)\equiv \sigma_0$ is constant. Then we can compute
\begin{align*}
X_t-X_0=\int_0^t \partial_s X_s~\mathrm{d}s\overset{\eqref{eq:simpleform}}{=}\int_0^t b(s,X_s)~\mathrm{d}s+\sigma_0 B_t.
\end{align*}
This equation at least makes sense but we don't know whether we can solve it.
If $\sigma$ is not constant, a term $\int_0^t\sigma(s,X_s)\xi_s~\mathrm{d}s$ appears.
In order to give sense to this integral, we will introduce It\^o-integrals or rough integrals.

\section*{Literature}
Schilling/Partsch: \glqq Brownian Motion\grqq
